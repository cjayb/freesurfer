% $Id: isxavg-fe.tex,v 1.1 2005/05/04 17:00:48 greve Exp $
\documentclass[10pt]{article}
\usepackage{amsmath}
%\usepackage{draftcopy}

%%%%%%%%%% set margins %%%%%%%%%%%%%
\addtolength{\textwidth}{1in}
\addtolength{\oddsidemargin}{-0.5in}
\addtolength{\textheight}{.75in}
\addtolength{\topmargin}{-.50in}

%%%%%%%%%%%%%%%%%%%%%%%%%%%%%%%%%%%%%%%%%%%%%%%%%%%%%%%%%%%%%%%%%
%%%%%%%%%%%%%%%%%%%%%%% begin document %%%%%%%%%%%%%%%%%%%%%%%%%%
%%%%%%%%%%%%%%%%%%%%%%%%%%%%%%%%%%%%%%%%%%%%%%%%%%%%%%%%%%%%%%%%%
\begin{document}

\begin{Large}
\noindent {\bf isxavg-fe} \\
\end{Large}

\noindent 
\begin{verbatim}
Comments or questions: analysis-bugs@nmr.mgh.harvard.edu\\
$Id: isxavg-fe.tex,v 1.1 2005/05/04 17:00:48 greve Exp $
\end{verbatim}

\section{Introduction}
{\bf isxavg-fe} is a program for intersubject averaging using a {\em
fixed-effects model} in which all the data from all subjects are
treated as if they came from a single subject (see isxavg-re for
random effects model).  It operates on the output of {\em selxavg},
and produces the same set of files.  Requires matlab 5.2 or higher.\\

In the within-subject analysis (selxavg), the parameters for each
subject $i$ are estimated (assuming white noise) by
\begin{equation}
\beta = (X'X)^{-1}X'y,
\end{equation}
where $y$ is the input data, and $X$ is the design matrix which codes
when each stimulus was presented as well as any assumption about the
event response. The covariance of $\beta$ (for noise of variance 1) is 
\begin{equation}
CoVar(\beta) = \Sigma_{\beta} = (X'X)^{-1}
\end{equation}
The variance of the residual error is estimated from
\begin{equation}
Var(e_r) = \hat{\sigma_{n}^2} = Var(y-X\beta) = Var(y - (X'X)^{-1}X'y)
\end{equation}

In the fixed-effects model, these three quantities are computed for a
group of subjects/sessions from those of the individuals:
\begin{equation}
\beta_{ffx} = \frac{1}{N_s} \sum \beta_{i}
\end{equation}
\begin{equation}
CoVar(\beta_{ffx}) = \Sigma_{\beta_{ffx}} = (\sum \Sigma_{\beta_i}^{-1})^{-1}
\end{equation}
\begin{equation}
\hat{\sigma_{n}^2} = \frac{\sum{\sigma_{n_i}^2 DOF_i}} 
                          {\sum{DOF_i}}
\end{equation}



\section{Usage}
Typing isxavg-fe at the command-line without any options will give the
following message:\\ 

\begin{small}
\begin{verbatim}
USAGE: isxavg-fe [-options] -i instem1 -i instem2 <...> -o outstem
   instem1   - prefix of first  .bfloat selxavg input files
   instem2   - prefix of second .bfloat selxavg input files
   ... 
   outstem   - prefix of .bfloat isxavg-fe output files
Options:
   -firstslice <int>  : first slice to process <0>
   -nslices <int>     : number of slices to process <auto>
   -umask <mask>      : set umask before running
   -monly mfile       : dont run, just create a matlab script file
\end{verbatim}
\end{small}

\section{Command-line Arguments}

\noindent
{\bf -i instem1}: stem of the volume in which the results of the
selective averaging for the first subject have been stored (see
selxavg).  There must be at least two input volumes, each preceded by
a {\em -i} flag.\\

\noindent
{\bf -o outstem}: stem of the output volume.  The format will be the
same as the input volumes, and isxavg-fe will also produce appropriate
dof, dat, and hdr files.\\

\noindent
{\bf -firstslice int}: first {\em anatomical} slice to process (usually 0).
This should not be confused with the first {\em functional} slice.\\

\noindent
{\bf -nslices int}: total number of {\em anatomical} slices to process.\\

\noindent
{\bf -umask}: specify the unix file access permissions with which the
output files and directories will be created.  Using {\em -umask 0}
will instruct isavg-fe to create files that are world-writable.  This
can be convenient when collaborating with other users.\\

\noindent
{\bf -monly}: only generate the matlab file which would accomplish the
analysis but do not actually execute it.  This is mainly good for
debugging purposes.\\

\end{document}
