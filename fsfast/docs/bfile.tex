% $Id: bfile.tex,v 1.1 2005/05/04 17:00:48 greve Exp $
\documentclass[10pt]{article}
\usepackage{amsmath}

%%%%%%%%%% set margins %%%%%%%%%%%%%
\addtolength{\textwidth}{1in}
\addtolength{\oddsidemargin}{-0.5in}
\addtolength{\textheight}{.75in}
\addtolength{\topmargin}{-.50in}

%%%%%%%%%%%%%%%%%%%%%%%%%%%%%%%%%%%%%%%%%%%%%%%%%%%%%%%%%%%%%%%%%
%%%%%%%%%%%%%%%%%%%%%%% begin document %%%%%%%%%%%%%%%%%%%%%%%%%%
%%%%%%%%%%%%%%%%%%%%%%%%%%%%%%%%%%%%%%%%%%%%%%%%%%%%%%%%%%%%%%%%%
\begin{document}

\begin{Large}
\noindent {\bf bvolume format} \\
\end{Large}

\noindent 
\begin{verbatim}
Comments or questions: analysis-bugs@nmr.mgh.harvard.edu
$Id: bfile.tex,v 1.1 2005/05/04 17:00:48 greve Exp $
\end{verbatim}

The {\bf bvolume format} is a format for saving and naming volumes of
MRI data in which each anatomical slice is stored in its own binary
file.  If there are multiple planes or time-points associated with the
anatomical slice (eg, as in an fMRI slice), then they are stored in
the binary file as well.  The files in the volume are specified by
three parameters: a stem, the number of slices, and type of bvolume.
The name of the file for a slice is constructed from these parameters
in the following way: {\em stem\_XXX.btype}, $stem$ is the stem, $XXX$
is the three digit slice number, and $btype$ is the type of bvolume.
If the slice number does not have three digits then zeros are used to
pad the slice number to three digits.  The type can either be $bshort$
or $bfloat$.  The data are stored in 16 bit signed integer format in
bshort files, whereas the data are stored in 32 bit floating point
format in bfloat files.

A text header file is associated with every binary slice file; its
name takes the form {\em stem\_XXX.hdr}.  The header file is a text
file with four numbers which designate how the data in the binary
bvolume are stored: (1) $N_{rows}$, number of rows, (2) $N_{cols}$,
number of columns, (3) $N_{tp}$, number of time-points (or planes or
frames or TRs), and (4) endianness.  The endianness is either 0
(big-endian) or 1 (little-endian).  The first $N_{cols}$ elements make
the first row.  The next $N_{cols}$ elements make the second row, etc,
until the entire slice is accounted for after which the next time
point starts.

The total number of values in a binary slice is equal to $N_v = N_{cols} *
N_{rows} * N_{tp}$.  The number of bytes in a bshort file equals
$2*N_v$ and $4*N_v$ for a bfloat file.  The number of elements in a
plane or time-point is $N_p = N_{cols} * N_{rows}$.  

There's also another header filed called the bhdr file that
accompanies a bvolume. Its name is stem.bhdr. Geometry information is
stored in this file.


\end{document}


