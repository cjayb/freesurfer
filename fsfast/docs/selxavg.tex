% $Id: selxavg.tex,v 1.1 2005/05/04 17:00:49 greve Exp $
\documentclass[10pt]{article}
\usepackage{amsmath}
%\usepackage{draftcopy}

%%%%%%%%%% set margins %%%%%%%%%%%%%
\addtolength{\textwidth}{1in}
\addtolength{\oddsidemargin}{-0.5in}
\addtolength{\textheight}{.75in}
\addtolength{\topmargin}{-.50in}

%%%%%%%%%%%%%%%%%%%%%%%%%%%%%%%%%%%%%%%%%%%%%%%%%%%%%%%%%%%%%%%%%
%%%%%%%%%%%%%%%%%%%%%%% begin document %%%%%%%%%%%%%%%%%%%%%%%%%%
%%%%%%%%%%%%%%%%%%%%%%%%%%%%%%%%%%%%%%%%%%%%%%%%%%%%%%%%%%%%%%%%%
\begin{document}

\begin{Large}
\noindent {\bf selxavg} \\
\end{Large}

\noindent 
\begin{verbatim}
Comments or questions: analysis-bugs@nmr.mgh.harvard.edu\\
$Id: selxavg.tex,v 1.1 2005/05/04 17:00:49 greve Exp $
\end{verbatim}

\section{Introduction}
{\bf selxavg2} is a program for performing selective averaging and
deconvolution of event-related fMRI images for a single subject over
multiple runs. It can also average data from blocked designs. Requires
matlab 5.2 or higher.\\

\section{Usage}
Typing selxavg2 at the command-line without any options will give the
following message:\\ 

\begin{small}
\begin{verbatim}
USAGE: selxavg2 [-options] -i instem1 -p parfile 1 [-i instem2 -p parfile2 ...] -o outstem
   instem1   - prefix of input file for 1st run
   parfile1  - path of parfile for 1st run
   [-tpexclfile file1] - file with timepoints of 1st run to exclude
   [instem2] - prefix of input file for 2nd run
   [parfile2]- path of parfile for 2st run
   [-tpexclfile file2] - file with timepoints of 2nd run to exclude
   outstem   - prefix of .bfloat output files
      NOTE: you must specify all runs to process now
Options:
   -TR <float>           : temporal scanning resolution in seconds 
   -TER <float>          : temporal estimation resolution <TR>
   -timewindow <float>   : length of hemodynamic response in seconds <20>
   -prestim <float>      : part of timewindow before stimulus onset (sec)
   -nobaseline           : do not remove baseline offset 
   -detrend              : remove baseline offset and temporal trend
   -rescale <float>      : rescale target (must run inorm first)
   -nskip <int>          : skip the first n data points in all runs
   -hanrad <float>       : radius of Hanning spatial filter (must be >= 1)
   -fwhm  <float>        : full-width/half-max of in-plane smoother (mm)
   -ipr   <float>        : in-plane resolution (mm)
   -gammafit delta tau   : fit HDR amplitude to a gamma function 
   -timeoffset <float>   : offset added to time in par-file in seconds <0>
   -nullcondid <int>     : Number given to the null condition <0>
   -firstslice <int>     : first slice to process <auto>
   -nslices <int>        : number of slices to process <auto>
   -percent pscstem      : compute and save percent signal change in pscstem
   -taumax  float        : don't assume white noise (spec max delay)
   -ecovmtx  stem        : compute and save residual error covariance
   -ecovmsn  stem        : residual error covariance, with spat norm
   -whtmtx   bfile       : whitening matrix
   -basegment            : force segmentation brain and air (with nonwhite)
   -mail                 : send mail to user when finished
   -eresdir dir          : directory in which to save eres vols
   -signaldir  dir       : directory in which to save signal vols
   -monly mfile          : don't run, just generate a matlab file
   -synth seed           : synthesize data with white noise (seed = -1 for auto)
   -parname name         : use parname in each input directory for paradigm file
   -cfg file             : specify a configuration file
   -umask umask          : set unix file permission mask
   -version              : print version and exit
\end{verbatim}
\end{small}

\section{Command-line Arguments}

Note that command-line arguments can also be specified inside of a
configuration file (see the -cfg argument).

\noindent
{\bf -i instem}: stem of the input functional volume for a single run.
It is assumed that the data are stored in {\em bfile format}.  If
multiple runs are to be analyzed, each input stem must be preceded by
a {\em -i} flag.\\

\noindent
{\bf -p parfile}: name of file in which the stimulus sequence is
stored (in {\em paradigm file format}). The ID for fixation (or
null-stimulus) defaults to 0 but can be specified using the {\em
-nullconid}.  The ID's must be consecutive integers (no skipping).  If
multiple runs are to be analyzed, a paradigm file must be specified
for each run.  A paradigm file will be paired with an input stem based
on order of appearance on the command-line.\\

\noindent
{\bf -tpexclfile}: (optional) full path of file that specifies the
time-points to ignore during an analysis.  The format is a single
columns of times within the run to exclude.  One may want to exclude
data points where the scanner has spiked.\\

\noindent
{\bf -o outstem}: this is the stem of the output volume (in bfile format)
to be generated.  The output consists of two volumes, one with stem
{\em oustem} in which the selective averages and standard deviations
are stored.  The other has a stem {\em oustem-offset} in which the
offset (or baseline) of each voxel is stored.  In addition, there is
a file called {\em outstem.dat} in which various parameters of the
selective averaging are stored.

\noindent
{\bf -TR float}: temporal sampling resolution (ie, time between scans
in seconds). \\

\noindent
{\bf -TER float}: temporal resolution (in seconds) at which to
estimate the hemodynamic response.  Defaults to the TR.  Note that
using anything besides the TR for the TER requires that the stimulus
presentation scheduling to have taken a finer TER into account.  This
cannot be done post-hoc. See {\bf optseq}.\\

\noindent
{\bf -timewindow float}: the amount of time needed to capture the
hemodynamic response as if a stimulus were presented in isolation.
This must include the prestimulus baseline period (see prestim).\\

\noindent
{\bf -prestim tPreStim}: specify a portion of {\em timewindow} to
estimate the hemodynamic response {\em before} the stimulus is
presented.  This can be used as a ``sanity check'' to verify that the
subject is not anticipating the stimulus. It can also be used to
remove baseline shifts when computing statistics.\\

\noindent
{\bf -nobaseline}: instruct selxavg NOT to remove the baseline
offset. By default, it will remove the offset on a run-by-run basis
from the raw fMRI data. Note that the offset will be stored in {\em
outstem-offset}. \\

\noindent
{\bf -detrend}: instruct selxavg to fit and remove any linear temporal
trend from the raw fMRI data.  This is done on a run-by-run basis.
Specifying this option forces the {\bf -baseline} option to be
specified.\\

\noindent
{\bf -rescale target}: this option used to be obsolete but is now
viable again when used in conjunction with {\bf inorm}.  {\bf inorm},
the global intensity normalization program, will produce a file called
{\em instem.meanval} in which the global mean value (after
segmentation) is stored.  When the -rescale option is specified,
selxavg will read the meanval file and the target value and rescale
the entire input volume so that it's new global mean is target.\\

\noindent
{\bf -nskip nSkip}: specify the number of {\em functional} slices to
skip at the beginning of each run.  This applies to all runs and
all anatomical slices. This functionality can also be implemented with
a tpexclude file.\\

\noindent
{\bf -fwhm FWHM}: specify the radius of a 2-D Hanning kernel
with which to spatially filter each functional image. The radius is in
mm. This is identical to -hanrad except that in -hanrad the radius is
measured in in-plane voxels.\\

\noindent
{\bf -hanrad HanRadius}: specify the radius of a 2-D Hanning kernel
with which to spatially filter each functional image.  The larger
the Hanning radius, the more smoothing. Radii less than 1.0 have no
effect.\\

\noindent
{\bf -gammafit $\Delta$ $\tau$}: fit the amplitude of a gamma function
to each hemodynamic response.  The output file will then hold the
best-fit amplitude instead of the hemodynamic response.  One can still
run {\em stxgrinder} on the output, but the {\em -ircorr} and {\em
delrange} options cannot be used nor can the {\em tm} or {\em Fm}
tests be specified.  Typical values: $\Delta = 2.25s$, $\tau = 1.25s$.
This option is needed for processing data blocked designs.
The gamma function has the form:
\begin{equation}
h(t) = 
\begin{cases}
0 & t < \Delta \\
(\frac{\tau e^2}{4})
( \frac{t-\Delta}{\tau} )^2 e^{{-(\frac{t-\Delta}{\tau}})} & t > \Delta \\
\end{cases}
\label{hideal.eqn}
\end{equation}

\noindent
{\bf -timeoffset float}: time (in seconds) to add to times in parfile
for cases when the acquisition and presentation sequences did not
start at the same time.  Positive offsets indicate that the stimulus
presentations began {\em after} the first scan.\\

\noindent
{\bf -firstslice int}: first {\em anatomical} slice to process (usually 0).
If unspecified, {\bf selxavg} will autodetect the first slice.  It is
highly recommended that the first slice number be 0.\\

\noindent
{\bf -nslices int}: total number of {\em anatomical} slices to
process.  If unspecified, {\bf selxavg} will autodetect the number of
slices.\\

\noindent
{\bf -percent pscstem}: instruct selxavg to compute and save the
hemodynamic response averages as a percentage of the baseline. The
results are stored in a separate volume with stem {\em pscstem}.  The
variances are scaled accordingly. Note that this is somewhat redundant
because the offsets are automatically stored in {\em outstem-offset},
and the percent signal change can be computed from the offsets and the
hemodynamic averages when viewing (eg, in {\bf yakview}).\\

\noindent
{\bf -taumax maxdelay}: {\em note: this feature is still a little
buggy, so use for evaluation only}. Specifies that selxavg should
compensate for temporal correlations in the noise by estimating and
fitting the temporal noise autocorrelation. The argument maxdelay is
the maximum delay (in seconds) that should be used when fitting the
noise autocorrelation function. It typically ranges between 20-30
sec.\\

\noindent
{\bf -basegment}: this flag forces selxavg to segment the brain from
the air in each functional run and estimate the noise autocorrelation
function only from those voxels labeled as brain.  It is highly
recommended that this flag be set.\\

\noindent
{\bf -eresdir}: specify a directory in which to save the residual
error for each slice and run.  Not very useful for most users.\\

\noindent
{\bf -signaldir}: specify a directory in which to save the estimated
signal for each slice and run. Not very useful for most users.\\

\noindent
{\bf -synth seed}: synthesize input data as white gaussian noise.  If
the seed is set to -1, then sexavg will generate a seed based on the
system clock.  This option is used for debugging.\\

\noindent
{\bf -monly}: only generate the matlab file which would accomplish the
analysis but do not actually execute it.  This is mainly good for
debugging purposes.\\

\noindent
{\bf -mail}: mail user upon completion of analysis.\\

\noindent
{\bf -cfg file}: specify a selxavg configuration file.  This is a file
with a different command-line option (and its arguments) on each line.
For example,
\begin{verbatim}
-TR 2
-hanrad 1.5
-detrend
-gammafit 2.7 1.9
\end{verbatim}
This is often more convenient typing all the options on the
command-line. Any command-line options specified with -cfg will
override any conflicting values in the configuration file.


\section{Output}

{\bf selxavg} will create several output files per anatomical slice
plus one file for all slices.  The names of these files are based on
the output stem provided on the command-line.  The hemodynamic
averages are stored in {\em outstem\_sss.bfloat} which has the
corresponding header file {\em outstem\_sss.hdr} (where {\em sss}
indicates the anatomical slice number).  A file called {\em
outstem-offset\_sss.bfloat} is created in which the offset (or
baseline) is stored. A file called {\em outstem\_.dat} is created into
which various parameters are stored (mostly command-line parameters).

The waveforms within the hemodynamic average file are stored in a
certain format (same as {\em selavg} format).  Each HDR estimate for
each condition will have $N_h$ components, where $N_h$ is the integer
ratio of the timewindow and the TR.  In a {\em bfloat} file, the data
are stored as image planes with each plane having a number of rows and
columns equal to the those of the original functional scan.  Each
condition will occupy $2 N_h$ planes for a total of $2 N_h N_c $
planes where $N_c$ is the total number of conditions including
fixation.  For a given condition, the first $N_h$ planes are the
averages for each point in the timewindow.  The next $N_h$ planes are
the standard deviations for each corresponding point in the
timewindow.  With {\em selxavg}, the averages for the fixation
condition are {\em always} set to zero; they are included in the file
strictly for backwards compatibility with {\em selavg} format.  The
standard deviation planes for the fixation condition are all equal to
the standard deviation of the residual error and are used in further
processes steps (eg, {\em stxgrinder}).  Again, this redundant
information is included for backwards compatibility.

\section{Examples}

Consider the case in which the current directory is {\em
/space/raid/4/users/inverse/fspace/950818SA}, the base directory of a
functional session.  Below this directory is, among others, a
subdirectory called {\em image}.  Below the {\em image} subdirectory
are five subdirectories: {\em r01, r02, r03, r04, r05}, one for each
of five runs.  The functional data are stored in the run
subdirectories.  For example, the functional data for anatomical slice
10 of run 2 would be in a file called {\em s\_010.bshort} under the
{\em r02} directory.  There would be a header file companion called
{\em s\_010.hdr}.  Assume that there are 16 anatomical slices, 0
through 15, and that the TR is 2 seconds.  Assume that the paradigm
files for the five runs are stored in a subdirectory of {\em image}
called {\em parfiles} and that the five paradigm files are named {\em
par01.dat, par02.dat, par03.dat, par04.dat, par05.dat}.  Assume also
that there are three types of stimuli presented in the experiment, one
of which is fixation.

To selectively average all the slices for the first run only, one
would execute the following command line:
\begin{verbatim}
selxavg -TR 2 -timewindow 24 -prestim 4 -detrend \
        -i image/r01/s -p parfiles/par01.dat \
        -o sxa1/havg 
\end{verbatim}

The {\em -TR 2} option tells selxavg to use a TR of 2 seconds.  The
{\em -timewindow 24} and {\em -prestim 4} indicate that the
hemodynamic response will be estimated 24 seconds, 4 seconds of which
will be before stimulus onset.  The {\em -detrend} option forces the
baseline and temporal trends to be removed. The input stem is {\em
image/r01/s} and the paradigm file is {\em parfiles/par01.dat}. The
output stem is {\em sxa1/havg}.  After successfully running selxavg
with these options, a new subdirectory will have appeared under the
{\em image} directory called {\em sxa1}. Within {\em sxa1}, there
should be 16 files of the form {\em havg\_sss.bfloat} (along with 16
header (.hdr), as well as 16 files of the form {\em
havg-offset\_sss.bfloat} (along with 16 header (.hdr), and one file
called havg.dat.

To average all five runs, execute:
\begin{verbatim}
selxavg -TR 2 -timewindow 20 -detrend \
        -i image/r01/s -p parfiles/par01.dat \
        -i image/r02/s -p parfiles/par02.dat \
        -i image/r03/s -p parfiles/par03.dat \
        -i image/r04/s -p parfiles/par04.dat \
        -i image/r05/s -p parfiles/par05.dat \
        -o sxa1/havg 
\end{verbatim}

Note that it is {\em not} possible to execute {\em selxavg} five different
times to average all the runs.  It must be executed once with all
five runs specified at the time of execution.

\section{Theory of Operation}

\subsection{Forward Model}

\begin{equation}
y = X_1 * h_1 + X_2 * h_2 + b + a t\ldots,
\end{equation}
where $y$ is the $N_t \times N_v$ matrix of raw data, $X_i$ is the
design matrix (ie, presentation shedule) for stimulus type $i$, $h_i$
is the (FIR) impulse response to stimulus type $i$, $b$ is the
baseline offset, $a$ is the slope of the linear drift, and $t$ is
time. This can all be collapsed into one matrix equation:
\begin{equation}
y = X * h  
\end{equation}
where $y$ is still the raw data, $X$ contains the presentation
schedules for all stimuli plus a column of ones for the baseline
offset plus a column of linearly increasing numbers for the temporal
trend, $h$ contains the (FIR) impulse response to all stimulus types
as well as the baseline offset and linear slope.

\subsection{Inverse Model}

In the equation above, we have $X$ and $y$ and want to
find $h$. This is accomplished by solving the inverse model:
\begin{equation}
\hat{h} = (X^T X)^{-1}X^T y
\end{equation}

\subsection{Construction of X}

For a given stimulus type, the number of rows of $X_i$ equal the
number of time points $N_t$, and the number of cols of $X_i$ equal the
number estimates in the hemodynamic response $N_h$. $N_h$, in turn, is
equal to the time window (as specified with the -timewindow option)
divided by the TER (as specified with the -TER option). The timewindow
should be an intetger multiple of the TER. If the TER is unspecified,
it defaults to the TR. If the prestimulus window is zero, $X_i$ is
constructed in the following way. When a stimulus of type $i$ is
presented, a 1 is placed in the first column of $X_i$ at the row
corresponding to the time point at which it was presnted. Another 1 is
then placed in the second column at the next row (ie, diagonally down
and to the right). This process is repeated to fill all the $N_h$
columns. 



\end{document}