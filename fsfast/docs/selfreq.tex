% $Id: selfreq.tex,v 1.3 2009/05/13 14:52:36 greve Exp $
\documentclass[10pt]{article}
\usepackage{amsmath}
%\usepackage{draftcopy}

%%%%%%%%%% set margins %%%%%%%%%%%%%
\addtolength{\textwidth}{1in}
\addtolength{\oddsidemargin}{-0.5in}
\addtolength{\textheight}{.75in}
\addtolength{\topmargin}{-.50in}

%%%%%%%%%%%%%%%%%%%%%%%%%%%%%%%%%%%%%%%%%%%%%%%%%%%%%%%%%%%%%%%%%
%%%%%%%%%%%%%%%%%%%%%%% begin document %%%%%%%%%%%%%%%%%%%%%%%%%%
%%%%%%%%%%%%%%%%%%%%%%%%%%%%%%%%%%%%%%%%%%%%%%%%%%%%%%%%%%%%%%%%%
\begin{document}

\begin{Large}
\noindent {\bf selfreqavg, selfreqcomb} \\
\end{Large}

\section{Introduction}
{\bf selfreqavg} and {\bf selfreqcomb} are programs for computing the
amplitude and phase of the hemodynamic response at a certain
frequency.  This is typically done in retinotopic (or phase encoding)
studies.  These programs are meant to replace {\bf analyse} and {\bf
combine} for phase encoding studies.  The paradigm consists of a
periodically varying stimulus (eg, a rotating spoke or
expanding/contracting ring).  The stimulus is characterized by its
fundamental frequency, direction, and type.  The the amplitude and
phase of hemodynamic response at the stimulation frequency are measured
by computing the FFT of the bold signal.  The significance of the
effect is assesed with an F-test in the following way. An F-ratio is
computed as the ratio of the signal power to the noise power. The
signal power is the power at the fundamental. The noise power is sum
of the remaining spectrum excluding the following components:
harmonics of the stimulus, components within one frequency bin of the
fundamental and harmonics, and the first three bins.

{\bf selfreqavg} computes the amplitude, phase, and noise directly
from the raw MR signal.  {\bf selfreqcomb} combines the results from
several invokations of {\bf selfreqavg} into a single average. This is
done by separately averaging the real and imaginary components of the
fundamental as well as averaging the noise power. The average real and
imaginary components are used to recompute the power at the
fundamental which is then used to perform the F-test as described
above. The software appropriately handles the case where the inputs
may have different degrees-of-freedom (eg, different number of time
points).\\


\section{Questions, Comments, and Bug Reports}
The software authors are interested in your questions, comments, and
bug reports.  However, since the software is free, our responsiveness
to your needs may be limitted.  To get the most out of an interaction
with us, please follow these suggestions: (1) read the documentation
thoroughly to make sure you are not asking about something for which
there is already an answer, (2) ask someone else in your
lab, (3) send COMPLETE information about the problem (ie, operating
system, machine type, user name, version number, log files, etc).
Send questions, comments, and bug reports to analysis-bugs\@nmr.mgh.harvard.edu.\\

\section{Usage}

Typing {\bf selfreqavg} at the command-line without any options will
give the following message:\\

\begin{small}
\begin{verbatim}
USAGE: selfreqavg
   -i instem  ...    : input stem(s)
   -TR TR            : temporal resolution (sec)
   -o outstem        : output stem
   -parname name     : parfile, same in each instem directory
   -stimtype  type   : eccen or polar (see also parname)
   -ncycles ncycles  : number of stimulation cycles (see also parname)
   -direction string : pos or neg (see also parname)
   -delay   delay    : global delay (in seconds)
   -sdf     file     : slice delay file
   -nskip nskip      : skip the first nskip TRs
   -rescale target   : rescale so global mean equals target
   -firstslice fs    : first anatomical slice
   -nslices    ns    : number of anatomical slices
   -monly mfile      : just create a matlab file
   -umask umask      : set unix file permission mask
   -version          : print version and exit
\end{verbatim}
\end{small}

Typing {\bf selfreqcomb} at the command-line without any options will give the
following message:\\ 

\begin{small}
\begin{verbatim}
USAGE: selfreqcomb
   -i instem  ...    : input stem(s)
   -o outstem        : output stem
   -firstslice fs    : first anatomical slice
   -nslices    ns    : number of anatomical slices
   -monly mfile      : just create a matlab file
   -umask umask      : set unix file permission mask
   -version          : print version and exit
\end{verbatim}
\end{small}

\section{Command-line Arguments}

Note that command-line arguments can also be specified inside of a
configuration file (see the -cfg argument).

\noindent
{\bf -i instem}: stem of the input volume for a single run.
It is assumed that the data are stored in {\em bfile format}.  If
multiple runs are to be analyzed, each input stem must be preceded by
a {\em -i} flag.\\

\noindent
{\bf -parname parname}: name of file in which the stimulus characteristics
are stored. The format is
\begin{verbatim}
stimtype  eccen
direction pos
ncycles   8
\end{verbatim}
where {\em stimtype} is the type of stimulus, {\em direction} is the
direction of the stimulus (either {\em pos} or {\em neg}), and {\em
ncycles} is the number of cycles in the run.  These are redundant with
other command-line arguments to {\bf selfreqavg}.  If both are
specified, presidence will be given to the command-line arguments.  If
the parname option is specified, there must be a file with the
specified name in each of the input directories (ie, parname is only
specified once).\\

\noindent
{\bf -o outstem}: this is the stem of the output volume (in bfile format)
to be generated.  This will have 12 planes:
\begin{enumerate}
\item $\pm ||log10(sig)||$ - sign based on the sign of the imaginary component
\item $||log10(sig)|| sin(\phi)$ 
\item $||log10(sig)|| cos(\phi)$ 
\item $F$
\item $\sqrt(F) sin(\phi)$ 
\item $\sqrt(F) cos(\phi)$ 
\item Standard deviation of the noise.
\item Real part of the signal at the fundamental.
\item Imaginary part of the signal at the fundamental.
\item Phase ($\phi$) in radians.
\item Mean Image.
\item Trend Image.
\item Magnitude (sum of squares of real and imag parts).
\item Phase masked by $p<0.01$
\end{enumerate}
Note: when using the {\em paint} command to render the results on the
cortical surface, make sure to set the {\em -offset} option to one
less than the plane number.

\noindent
{\bf -TR float}: temporal sampling resolution (ie, time between scans
in seconds). \\

\noindent
{\bf -delay float}: specify the hemodynamic delay (in seconds) to
apply to all voxels.\\

\noindent
{\bf -sdf filename}: specify a slice delay file (SDF).  The SDF has
two columns.  The first is the slice number, and the second is
acquisition delay (in seconds) of the slice with respect to the start
of the TR.\\

\noindent
{\bf -detrend}: instruct selfreqavg to fit and remove any trend
from the raw fMRI data that is linearly related to time.
This is done on a run-by-run basis.\\

\noindent
{\bf -rescale target}: used in conjunction with {\bf inorm}.  {\bf inorm},
the global intensity normalization program, will produce a file called
{\em instem.meanval} in which the global mean value (after
segmentation) is stored.  When the -rescale option is specified,
selfreqavg will read the meanval file and the target value and rescale
the entire input volume so that it's new global mean is target.\\

\noindent
{\bf -nskip nSkip}: specify the number of {\em functional} slices to
skip at the beginning of each run.  This applies to all runs and
all anatomical slices.  Note: this can severely affect the FFT
computations. \\

\noindent
{\bf -firstslice int}: first {\em anatomical} slice to process (usually 0).
If unspecified, the first slice will be autodetected.  It is
highly recommended that the first slice number be 0.\\

\noindent
{\bf -nslices int}: total number of {\em anatomical} slices to
process.  If unspecified, the number of slices will be autodetected.\\

\noindent
{\bf -monly}: only generate the matlab file which would accomplish the
analysis but do not actually execute it.  This is mainly good for
debugging purposes.\\

\section{Interfacing with FreeSurfer and Paint}

For phase encoding studies, FreeSurfer expects the surface data to
conform to a certain naming convention.  Specifically, FreeSurfer
expects three surface files for each hemisphere.  For example, for the
left hemisphere these files are: sigf-lh.w, sig2-lh.w, and
sig3-lh.w. These correspond to the first three planes in the output of
{\bf selfreqavg}.  These .w files are created by paint.  For an output
stem of {\em sfaout}, paint would be invoked in the following way:
\begin{verbatim}
  paint sfaout_%03d.bfloat lh orig sigf-lh.w -nslices 16 -imageoffset 0 
  paint sfaout_%03d.bfloat lh orig sig2-lh.w -nslices 16 -imageoffset 1 
  paint sfaout_%03d.bfloat lh orig sig3-lh.w -nslices 16 -imageoffset 2 
\end{verbatim}
Note that before paint can be invoked, the subject must have been
anatomically reconstructed using the FreeSurfer tools, and the
functional run must have been registered with the anatomical run.

\section{Example}

Consider a set of raw data consisting of 6 runs. Runs 1, 3, and 5 use
a polar stimulus, where as runs 2, 4, and 6 use an eccentricity
stimulus. All have 8 cycles per run in the same (positive) direction.
The TR is 2 seconds, and 16 slices were collected.  The data are in
bshort format with stem {\em f} in the following directories: 001,
002, 003, 004, 005, 006.  To process the data, cd to the directory
just above the run directories, and run:
\begin{verbatim}
  selfreqavg -i 001/f -i 003/f -i 005/f -o polar/sfaout \
     -TR 2 -ncycles 8 -stimtype polar -direction pos 

  selfreqavg -i 002/f -i 004/f -i 006/f -o eccen/sfaout \
     -TR 2 -ncycles 8 -stimtype eccen -direction pos 
\end{verbatim}
These will create two new directories: {\em polar} and {\em eccen}
with bfloat volumes with stem {\em sfaout}.

To paint this data, again cd to the directory just above the run
directories, and type 
\begin{verbatim}
  paint polar/sfaout_%03d.bfloat lh orig polar/sigf-lh.w -nslices 16 -imageoffset 0 
  paint polar/sfaout_%03d.bfloat lh orig polar/sig2-lh.w -nslices 16 -imageoffset 1 
  paint polar/sfaout_%03d.bfloat lh orig polar/sig3-lh.w -nslices 16 -imageoffset 2 

  paint eccen/sfaout_%03d.bfloat lh orig eccen/sigf-lh.w -nslices 16 -imageoffset 0 
  paint eccen/sfaout_%03d.bfloat lh orig eccen/sig2-lh.w -nslices 16 -imageoffset 1 
  paint eccen/sfaout_%03d.bfloat lh orig eccen/sig3-lh.w -nslices 16 -imageoffset 2 
\end{verbatim}
Note that before paint can be invoked, the subject must have been
anatomically reconstructed using the FreeSurfer tools, and the
functional run must have been registered with the anatomical run.

To view the results on the surface, create and executre a surfing script with the
following content:
\begin{verbatim}
  #!/bin/csh -f
  set name = subject  # put the subject's name here
  set hemi = lh
  setenv eccendir eccen
  setenv polardir polar
  #setenv patch occip.patch.flat
  setenv revpolarflag 1
  setenv rgbname eccenthinmot
  setenv fthresh 0.4
  setenv fslope 1.3
  setenv fmid 0.8
  setenv angle_offset 0.77
  setenv noexit
  tksurfer -$name $hemi 1000a -tcl eccen-flat.tcl
\end{verbatim}


\section{mri\_glmfit and selfreq}

Testing multiple regressors across subjects is challenging in that
there are multiple measures per subject (usually there is only
one). One could test a magnitude, but then that does not have 0 mean.
Another is to estimate the real and imaginary components across
subject separately and then perform an multivariate (F) test on
those. This assumes that there is no correlation between the real/imag
within a subject. This is what is described below.


\begin{verbatim}

# Get real components
foreach session
  mri_convert h-lh.bhdr h-lh-real.mgh --frame 7 # grab real component
end
mri_concat allssessions/h-lh-real.mgh --o allsession.h-lh-real.mgh

# Get imaginary components
foreach session
  mri_convert h-lh.bhdr h-lh-imag.mgh --frame 8 # grab imaginary component
end
mri_concat allsessions/h-lh-imag.mgh --o allsession.h-lh-imag.mgh

# Now concat the real and imag
mri_concat allsession.h-lh-real.mgh allsession.h-lh-imag.mgh 
  --o allsession.h-lh-complex.mgh

\end{verbatim}

Create an fsgd file with your subjects divided into two groups, one
for real and one for imaginary. If you already have groups/classes,
then you will need to divide those into two. Take your normal fsgd
file with each subject listed once and concatenate the subject list
(ie, the Input lines). Put the first half in the real group and the
second half in the imaginary group. If you have variables
(covariates), then those stay the same for each subject. In the case
where you have only one group (dividied into real and imag), then the
first regressor will be the mean real part and the second will be the
mean imaginary part. Create the following contrast matrix (adding 0s
if you have covariates):
\begin{verbatim}
1 0
0 1
\end{verbatim}
This will test the magnitude=0.

If you have two groups (say Male and Female), then you'd create four
classes, eg:
\begin{verbatim}
Class RealMale
Class RealFemale
Class ImagMale
Class ImagFemale
\end{verbatim}

You would then create the following contrast matrix:
\begin{verbatim}
1 1 0 0
0 0 1 1
\end{verbatim}



\end{document}
