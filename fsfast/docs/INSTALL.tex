\documentclass[10pt]{article}
\usepackage{amsmath}
%\usepackage{draftcopy}

%%%%%%%%%% set margins %%%%%%%%%%%%%
\addtolength{\textwidth}{1in}
\addtolength{\oddsidemargin}{-0.5in}
\addtolength{\textheight}{.75in}
\addtolength{\topmargin}{-.50in}

%%%%%%%%%%%%%%%%%%%%%%%%%%%%%%%%%%%%%%%%%%%%%%%%%%%%%%%%%%%%%%%%%
%%%%%%%%%%%%%%%%%%%%%%% begin document %%%%%%%%%%%%%%%%%%%%%%%%%%
%%%%%%%%%%%%%%%%%%%%%%%%%%%%%%%%%%%%%%%%%%%%%%%%%%%%%%%%%%%%%%%%%
\begin{document}

\begin{center}
\begin{LARGE}
\noindent {\bf FreeSurfer Functional Analysis STream (FS-FAST)\\
Download and  Installation} \\
\end{LARGE}
\end{center}

\noindent 
\begin{verbatim}

This gives instructions on how to download, install, and configure the
environment for running FS-FAST, the fMRI analysis stream used by many
of the researchers in the MGH-NMR Center.  While designed to integrate
with the FreeSurfer anatomical analysis package, FS-FAST can also be
used without FreeSurfer.  The FS-FAST package does not include the
FreeSurfer software which can be downloaded from
www.nmr.mgh.harvard.edu/freesurfer.  For those in the Center, you do
not need to Download or Install, just follow the instructions under
Environment.

------------ Download --------------------
Download the most recent version
from ftp://ftp.nmr.mgh.harvard.edu/pub/flat/fmri-analysis

  % ftp ftp.nmr.mgh.harvard.edu
  user: anonymous
  passwd: your email

  ftp> cd pub/flat/fmri-analysis

  ftp> binary
  ftp> get fmri-YYMMDD.tar.gz

Where YYMMDD corresponds to the most recent date.

--------------- Install -------------------------
Move fmri-YYMMDD.tar.gz to the directory under which you want
the package to be installed.

Uncompress the tar file:
  % gunzip fmri-YYMMDD.tar.gz

This creates fmri-YYMMDD.tar which can be detarred:
  % tar xvf fmri-YYMMDD.tar

This will create a directory called fmri-YYMMDD under which there
are 4 subdirectories: bin  docs  src  toolbox.  You may want to
create a symbolic link to fmri-YYMMDD from fmri:
% ln -s fmri-YYMMDD fmri

---------------- Environment ----------------------------------------
In your .cshrc file (or equivalent), create a new environment variable:
  setenv FMRI_ANALYSIS_DIR name-of-install-directory

For those in the MGH-NMR center, you can set
  setenv FMRI_ANALYSIS_DIR /homes/nmrnew/home/inverse/fmri

Add $FMRI_ANALYSIS_DIR/bin and $FMRI_ANALYSIS_DIR/bin/`uname -s` 
directories to your path.

Add the toolbox directory to your matlab path.  This is done
by adding following lines to the startup.m file in your 
matlab directory (ie, ~/matlab/startup.m):

  fmri_analysis_dir = getenv('FMRI_ANALYSIS_DIR');
  fmritoolbox = sprintf('%s/toolbox',fmri_analysis_dir);
  path(path,fmritoolbox);

Note that you must have matlab 5.2 or higher to run the software.
The software has been tested under Linux, IRIX, and Sun.  The scripts
first look for a command called "matlab5" then for "matlab".  If
it does not find either of those, it will exit.

To use the motion correction, you must have AFNI installed and its
binaries in your path.  For those in the MGH-NMR Center, you can
add the following to your path:

/homes/nmrnew/home/inverse/afni/`uname -s`

---------------- Getting Started ---------------------------------- 
The file called overview.ps (or overview.tex) in the docs directory
has information about what can be done with the software.  There's
also a lot more documentation in the docs as to how each program
works.

---------------- Bugs, Comments, Questions ----------------------------- 
Send comments or questions to anlysis-bugs@nmr.mgh.harvard.edu.
When making bug reports, please follow these steps:
1.  Specify what type of machine you were on when the problem occured.
2.  Specify what program (along with command-line options) you were
running when the  problem occured.  NOTE: "program" means a program from the
FAST package, not one that you have created that calls FAST programs.
3.  Specify the directory you were in when you ran the program. 
4.  Include any text that the program printed out. 
5.  Include the log file (if it exists).

---------------- Dislaimer ------------------------
See docs/DISCLAIMER

---------------- Copywrite ------------------------
See docs/COPYWRITE
\end{verbatim}

\end{document}