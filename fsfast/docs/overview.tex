% $Id: overview.tex,v 1.1 2005/05/04 17:00:49 greve Exp $
\documentclass[10pt]{article}
\usepackage{amsmath}
%\usepackage{draftcopy}

%%%%%%%%%% set margins %%%%%%%%%%%%%
\addtolength{\textwidth}{1in}
\addtolength{\oddsidemargin}{-0.5in}
\addtolength{\textheight}{.75in}
\addtolength{\topmargin}{-.50in}

%%%%%%%%%%%%%%%%%%%%%%%%%%%%%%%%%%%%%%%%%%%%%%%%%%%%%%%%%%%%%%%%%
%%%%%%%%%%%%%%%%%%%%%%% begin document %%%%%%%%%%%%%%%%%%%%%%%%%%
%%%%%%%%%%%%%%%%%%%%%%%%%%%%%%%%%%%%%%%%%%%%%%%%%%%%%%%%%%%%%%%%%
\begin{document}

\begin{center}
{\LARGE The NMR-MGH fMRI Processing Stream}
\end{center}

This document gives a short overview of the functional MRI processing
used at the NMR Center at Massachusettes General Hospital.  Pointers
to other documents are indicated in {\em italics}.  Please send
questions or comments to {\bf analysis-bugs@nmr.mgh.harvar.edu}.  The
package is available via ftp from {\bf
ftp://ftp.nmr.mgh.harvard.edu/pub/flat/fmri-analysis}.

\section{Summary}

The NMR-MGH fMRI processing stream is based on selective averaging
(Dale and Buckner, 1997) and deconvolution (Buroc, et al, 1998).  The
method models the BOLD signal as a linear combination of
time-invariant hemodynamic responses imbedded in Gaussian noise.  For
event-related designs, the hemodynamic responses to different event
types are modeled as a sum of delayed delta functions (regressors)
with unknown weights (regression coefficients). No assumptions about
the form of the hemodynamic shape (delay or dispersion) are made.  For
block designs, the shape of the hemodynamic response (ie, the
regressor) is assumed but its weight is unknown.  In both cases, the
regressors are estimated using a least-squares method (Hamilton,
1994).  For event-related deisgns, this method results the optimal
{\em unbiased} estimator of the hemodynamic response.  The covariance
matrix of the regression coefficients is maintained and used to
compute multivariate statistics.  Individual subjects can be combined
using either a random-effects or fixed-effects model (after
appropriate spatial normalization).  The package includes a tool for
aiding in the design of rapid-presentation event related designs. The
package also includes visualization tools for overlaying functional
activation on structural images as well as to view the hemodynamic
responses for different conditions in a ``point-and-click''
environment.\\

\section{Computational Requirements}

The software will run under Linux, Sun, and SGI operating systems.
$Matlab^{TM}$ 5.2 or higher is required to run the software.  Note,
however, knowledge of matlab is not required as all of the programs
run directly from the unix command-line.  Note: the motion correction
program is a front-end for running the AFNI motion correction
(see http://varda.biophysics.mcw.edu/~cox/ for more information about
AFNI). You must have the AFNI software package installed on your
computer and the binaries in your path.  \\

\section{File Formats}

Input images are assumed to be stored in {\em bfile} format.  The
{\em bfile} format divides each volume into a number of files, one for
each anatomical slice.  Each slice may contain multiple planes
corresponding to the slice image at different times.  Each slice file
has a name conforming to the format stem\_XXX.btype, where ``XXX'' is
the three-digit, zero padded slice number, and ``btype'' is either
``bshort'' or ``bfloat''.  Bshorts store signed 16 bit integers.
Bfloats store 32 bit floating point numbers.  Each slice has a
cooresponding header file with a name conforming to the format
``stem\_XXX.hdr''.  This header has four numbers: number of columns,
number of rows, number of planes, and endianness. The endianness is
either 0 (big-endian) or 1 (little-endian).

The scheduling of stimuli is indicated in the {\em paradigm} file.
This is a file with (at least) two columns of numbers.  The first
column indicates the time at which a stimulus was presented, and the
second indicates the type of stimulus that was presented at that
time. The stimulus types are defined by this file.

\section{Rapid-Presenation Event-Related Experimental Design}

Rapid-Presenation Event-Related (RPER) is a type of fMRI experimental
design where the stimuli are presented closely enough in time that
their hemodynamic responses can overlap.  In contrast, fixed-interval
event-related (FIER) designs require that stimulus presentations be
separted enough in time that the hemodynamic responses do not overlap.
RPER can result in experiments with signficantly more statistical
power then FIER (Dale, et al, 1999).  RPER allows for better
randomization of events.  Analysis of RPER does make the assumption
that the overlap is linear.  The validity of linear overlap has been
explored in several experiments (eg, Dale and Buckner, 1997; Miezin,
et al, 1999).

The scheduling of events in RPER is a non-trivial task because events
can occur at any time during the run.  However, different schedules
may result in different statistical power.  Accordingly, we have
devised a technique to search for sequences that will offer the most
power.  See {\em optseq}.


\section{Processing Stream}

Individual Subject Processing Stream Summary:\\
\begin{enumerate}
\item Motion Correction (mc-afni)
\item Intesity Normalization (inorm)
\item Selective Averaging (selxavg)
\item Statistical Analysis (stxgrinder)
\item Visualization (yakview)
\end{enumerate}

Group Processing Stream Summary:\\
\begin{enumerate}
\item Motion Correction (mc-afni)
\item Intesity Normalization (inorm)
\item Selective Averaging (selxavg)
\item Spatial Normalization, eg Talairach
\item Inter-Subject Averaging, Fixed (isxavg-fe) and Random
(isxavg-fe) Effects.
\item Visualization (yakview)
\end{enumerate}

\section{The Sessions Environment}

All of the functional processing can be performed on data arbitrary
file names in arbitrary directories.  However, it can be a tedious
process to specify all of these command-line parameters.  To
allieviate this problem, we have developed a file and directory naming
structure that we call the {\em Sessions Environment}.  Commands that
exploit this enviroment typically end with {\em -sess} suffix.


\noindent
{\bf References:}\\

\indent Burock, M. A. (1998). Design and statistical analysis of fMRI
experiments to assess human brain hemodynamic responses. Unpublished
Master's Thesis, M.I.T.\\

\indent Burock, M. A., Buckner, R. L., Woldorff, M. G., Rosen, B. R.,
\& Dale, A. M. (1998). Randomized event-related experimental designs
allow for extremely rapid presentation rates using functional MRI.
Neuroreport, 9, 37 35-3739.\\

\indent Buroc, M.A. and Dale, A.M. (2000) Estimation and Detection of
Event-Related fMRI Signals with Temporally Correlated Noise: A
Statistically Efficient and Unbiased Approach. Human Brain Mapping
11:249-260.\\

\indent Dale, A.M., Greve, D.N., and Burock, M.A. (1999) Optimal Stimulus
Sequences for Event-Related fMRI.  5th International Conference on
Functional Mapping of the Human Brain. Duesseldorf, Germany. June
11-16. \\

\indent Dale AM, and Buckner RL (1997). Selective averaging of rapidly
presented individual trials using fMRI. Hum Brain Mapping 5:329-340.\\

\indent Hamilton, James D. (1994).  Time Series Analysis. Princeton
University Press. Princeton, NJ.\\

\indent Miezin, F.M., Maccotta, L., Ollinger, J.M., Petersen, S.E.,
and Buckner, R.L., 1999. Characterizing the hemodynamic response:
Effects of presentation rate, sampling procedure, and the possibility
of ordering brain activity based on relative timing. Submitted to
NeuroImage.\\


\end{document}