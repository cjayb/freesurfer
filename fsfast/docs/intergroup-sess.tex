% $Id: intergroup-sess.tex,v 1.1 2005/05/04 17:00:48 greve Exp $
\documentclass[10pt]{article}
\usepackage{amsmath}
%\usepackage{draftcopy}

%%%%%%%%%% set margins %%%%%%%%%%%%%
\addtolength{\textwidth}{1in}
\addtolength{\oddsidemargin}{-0.5in}
\addtolength{\textheight}{.75in}
\addtolength{\topmargin}{-.50in}

%%%%%%%%%%%%%%%%%%%%%%%%%%%%%%%%%%%%%%%%%%%%%%%%%%%%%%%%%%%%%%%%%
%%%%%%%%%%%%%%%%%%%%%%% begin document %%%%%%%%%%%%%%%%%%%%%%%%%%
%%%%%%%%%%%%%%%%%%%%%%%%%%%%%%%%%%%%%%%%%%%%%%%%%%%%%%%%%%%%%%%%%
\begin{document}

\begin{Large}
\noindent {\bf intergroupavg-sess} \\
\end{Large}

\noindent 
\begin{verbatim}
Comments or questions: analysis-bugs@nmr.mgh.harvard.edu\\
$Id: intergroup-sess.tex,v 1.1 2005/05/04 17:00:48 greve Exp $
\end{verbatim}

\section{Introduction}

{\bf intergroupavg-sess} is a program for intergroup averaging. The
individual groups are assumed to have been averaged using a random
effects model (see {\bf isxavg-re-sess}).  Inter-group statistics are
computed based on the t-statistic:
\begin{equation}
t = \frac{a_1 - a_2} { \sqrt{\frac{s_1^2}{n_1} + \frac{s_2^2}{n_2} }}
\label{igt.eqn}
\end{equation}
where $a_i$ is the average from Group $i$, $s_i^2$ is the variance
measured from Group $i$, and $n_i$ is the number of subjects in Group
$i$. The degrees-of-freedom is $DOF = n_1+n_2-2$.  This formula is
used to asses the significance of the difference between the averages
of two groups with different and unknown variances.  The average and
variance for each group are computed according to a random effects
model.  To run the random effects averaging, the user chooses a
contrast which is used to collapse the hemodynamic responses across
time and condition to yield a single number for each subject.  The
group average will be the average of this number over all the
subjects. The group variance will be the variance of this number
across all the subjects in the group.

\section{Usage}
Typing intergroupavg-sess at the command-line without any options will
give the following message:\\

\begin{small}
\begin{verbatim}
USAGE: intergroupavg-sess
Options:
   -analysis analysisname : session-level functional analysis name
   -contrast contrastname : contrast used in the random effects avg
   -group1    groupname    : name of group1 (positive)
   -group2    groupname    : name of group2 (negative)
   -intergroup intergroupname : name of intergroup average
   -space    spacename    : space in which to average (native, tal, sph)
   -hemi     hemisphere   : with sph space <lh rh>
   -umask umask   : set unix file permission mask
   -version       : print version and exit
\end{verbatim}
\end{small}

\section{Command-line Arguments}

\noindent
{\bf -o analysis}: analysis created by {\em mkanalysis-sess} and used
with {\em selxavg-sess} on each individual session.\\

\noindent
{\bf -i contrast}: contrast created by {\em mkcontrast-sess} and used
when running {\em isxavg-re-sess}. If the contrast is multivariate,
then sig and minsig maps will be produced.\\

\noindent
{\bf -group1 groupname}: group name given when calling {\em
isxavg-re-sess}. This refers to the group that will be ``positive''
(ie, $a_1$ in Equation \ref{igt.eqn}).\\

\noindent
{\bf -group2 groupname}: group name given when calling {\em
isxavg-re-sess}. This refers to the group that will be ``negative''
(ie, $a_2$ in Equation \ref{igt.eqn}).\\

\noindent
{\bf -intergroup name}: Name given to the intergroup average. This
name is used in subsequent commands just as one would use the session
id or the group name (eg, to view results).\\

\noindent
{\bf -space spacename}: Space in which {\em isxavg-re-sess} was
run. This is the space that each individual was transformed to so that
all individuals across all groups are registered. Options are: tal (for
talairach), sph (for spherical surface-based), or the name of a
region-of-interest (created with func2roi-sess).\\

\noindent
{\bf -hemi hemisphere}: Specify a given hemisphere when using
spherical-based averaging (ie, {\em -space sph}). Options are lh and
rh.\\ 

\noindent
{\bf -umask mask}: unix file permission mask. Set to 0 to share files
with everyone.\\

\section{Example}

Say you have 10 subjects in 2 groups (A and B) with sessids:
subject1a, ..., subject5a, subject1b, ..., subject5b. Create sessid
files called a.sid and b.sid and sessdir file sess.dir. Let the
analysis name be ``main''. The contrast you are interested in is
``up-vs-down''. \\

\begin{enumerate}
\item Run selxavg-sess on all subjects: \\
selxavg-sess -sf a.sid -sf b.sid -df sess.dir -analysis main
\item Resample them into talairach space:\\
func2tal-sess -res 4 -sf a.sid -sf b.sid -df sess.dir -analysis main
\item Run random effects group averaging for each group:\\
isxavg-re-sess -group GroupA -sf a.sid -df sess.dir -analysis main
-contrast up-vs-down -space tal\\
isxavg-re-sess -group GroupB -sf b.sid -df sess.dir -analysis main
-contrast up-vs-down -space tal\\
\item Run intergroup averaging:\\
intergroupavg-sess -analysis main -contrast up-vs-down -group1 GroupA
-group2 GroupB -intergroup AMinusB -space tal\\
\item View the results on the talairach volume:\\
tkmedit-sess -analysis main -s AMinusB -isxavg random -contrast
up-vs-down  -space tal -d .\\
\end{enumerate}



\end{document}



